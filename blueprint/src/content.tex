% In this file you should put the actual content of the blueprint.
% It will be used both by the web and the print version.
% It should *not* include the \begin{document}
%
% If you want to split the blueprint content into several files then
% the current file can be a simple sequence of \input. Otherwise It
% can start with a \section or \chapter for instance.

\begin{definition}[約数関数]~\

自然数 \( n \) について、\( n \) の正の約数の総和を表す関数を約数関数といい、\( \sigma(n) \) と表す。

\( \sigma(n)\) は以下の性質をもつ。
\vskip.5\baselineskip
(1) \( n = 1 \) のときに限り、\( \sigma(n) = 1 \).

(2) \( n \) が完全数のときに限り、\( \sigma(n) = 2n \).

(3) \( n \) が素数のときに限り、\( \sigma(n) =  1 + n \).

(4) 乗法的関数である。すなわち、互いに素な自然数 \( m, n \) に対して、\( \sigma(mn) = \sigma(m) \cdot \sigma(n) \).

\end{definition}


\begin{definition}[完全数]~\

自然数 \( N \) が完全数であるとは、\( N \) の正の約数の総和が \( 2N \) に等しいことをいう。

すなわち、\( \sigma(N)\) = \( 2N \).

\end{definition}


\begin{theorem}[Euclid-Euler]~\
\lean{Theorems100.Nat.eq_two_pow_mul_prime_mersenne_of_even_perfect}

\( n \) を自然数とする。
メルセンヌ数 \( 2^n - 1 \) が素数であるとき、\( 2^{n-1}(2^n - 1) \) は完全数である。

逆に、任意の偶数の完全数は \( 2^{n-1}(2^n - 1) \) の形で表され、このとき \( 2^n - 1 \) は(メルセンヌ)素数である。

\end{theorem}


\begin{proof}

[十分性(Euclid)]

\( N = 2^{n-1}(2^n - 1) \) とする。(\(n \) は自然数). ただし、メルセンヌ数 \( 2^n - 1 \) は素数。

\( 2^{n-1} \) と \( 2^n - 1 \) は互いに素より、\( \sigma(N) = \sigma(2^{n-1}) \cdot \sigma(2^n - 1) \) が成り立つ。

このとき、
\begin{align*}
\sigma(N) &= \sigma(2^{n-1}) \cdot \sigma(2^n - 1) \\
          &= (1 + 2 + 2^2 + \ldots + 2^{n-1}) \cdot \{1 + (2^n - 1)\} \\
          &= (\frac{2^n-1}{2-1}) \cdot (2^n) \\
          &= 2^n(2^n - 1)
\end{align*}

したがって、\(\sigma(N) = 2 \cdot 2^{n-1}(2^n - 1) = 2N\) が成り立つため、\( N \) は完全数である。

\end{proof}


\begin{proof}

[必要性(Euler)]

任意の偶数の完全数 \( N \) は自然数 \( n \) と奇数 \( k \) を用いて、\( N = 2^{n-1}k \) と表せる。

\( N \) は完全数より、\(\sigma(N) = 2N = 2^{n}k\) が成り立つ。

また、\(2^{n-1}\) と \( k \) は互いに素より、\(N\) の約数の和は次のように分解できる:
\begin{align*}
\sigma(N) &= \sigma(2^{n-1}) \cdot \sigma(k) \\
          &= (2^n - 1) \cdot \sigma(k)
\end{align*}

\( \sigma(N) = 2^{n}k \) より、\( \sigma(k) = \frac{2^{n}k}{2^n - 1} = k + \frac{k}{2^n-1} \) が成り立つ。(\(\because \)  \(2^n - 1 \neq 0 \))

ここで、\(\sigma(k)\) と \(k\) は自然数より、\(\frac{k}{2^n-1}\) も自然数。

したがって、\(2^n - 1\) は \(k\) の約数である。

\(k = (2^n-1) \cdot m\) とおく。このとき、\( \sigma(k) = k + m \)。(\(m \) は自然数)

ここで、\( m \neq 1 \) とすると、\(k\) は少なくとも \(1, m, k \) を約数に持つため、
\(\sigma(k) \geq 1 + m + k\) が成り立つ。

しかし、これは \( \sigma(k) \geq 1 + k + m > k + m \) となり、矛盾。

よって、\( m = 1 \) であり、\( k = 2^n - 1 \) が成り立つ。

したがって、任意の偶数の完全数は \( N = 2^{n-1}(2^n - 1) \) の形で表される。

またこのとき、\(\sigma(k) = 1 + k \) となり、\( k = 2^n - 1 \) は素数である。

\end{proof}
